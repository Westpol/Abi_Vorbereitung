\documentclass[fontsize=12pt,paper=a4,DIV12,cleardoublepage=empty, 
liststotoc,idxtotoc,bibtotoc]{article}
\usepackage[ngerman]{babel}
\usepackage[utf8]{inputenc}
\usepackage[pdftex]{graphicx}
\usepackage{amsmath}
\usepackage{longtable}
\usepackage{stmaryrd}
\usepackage{colortbl}
\usepackage{eurosym}
\usepackage{amssymb}
\usepackage{amsthm}
\usepackage{nicefrac}
\usepackage{lscape}
\usepackage{pdfpages} 
%Seitenlayout
\renewcommand{\thesection}{\Roman{section}}
\renewcommand{\thesubsection}{\Roman{section}.\arabic{subsection}}
\usepackage[left=25mm, right=25mm, bottom=30mm]{geometry}
\usepackage[doublespacing]{setspace}
\usepackage[colorlinks=true, linkcolor=black, citecolor=black, urlcolor=blue]{hyperref} 
\usepackage[figure]{hypcap}
%Textstruktur
\usepackage{float}
\usepackage{wrapfig}
\usepackage{tabularx}
%Textaussehen
\usepackage{framed}
%Verzeichnisse
\usepackage{csquotes}
\usepackage[backend=biber, bibstyle=authoryear-ibid, uniquelist=false, maxbibnames=9, maxcitenames=3, citestyle=authoryear-icomp, isbn=true, url=true, block=space, pagetracker=page,   giveninits=false, dateabbrev=false, dashed=false ]{biblatex}
\addbibresource{MA_literatur.bib}
\DefineBibliographyStrings{ngerman}{ 
	andothers = {{et\,al\adddot}},             
} 
\usepackage{etoolbox}
\apptocmd{\UrlBreaks}{\do\f\do\m}{}{}
\setcounter{biburllcpenalty}{9000}% Kleinbuchstaben
\setcounter{biburlucpenalty}{9000}% Großbuchsta
\expandafter\def\expandafter\quote\expandafter{\quote\small\singlespacing}

\setcounter{section}{0}
\setcounter{subsection}{0}

\newcommand*{\meincite}[1]{\citeauthor{#1} (\citeyear{#1})}

\newcommand{\KK}{\mathbb{•}{K}}
\newcommand{\CC}{\mathbb{C}}
\newcommand{\RR}{\mathbb{R}}
\newcommand{\QQ}{\mathbb{Q}}
\newcommand{\ZZ}{\mathbb{Z}}
\newcommand{\NN}{\mathbb{N}}
\newcommand{\PPO}{\mathcal{P}(\Omega)}

\theoremstyle{plain}
\newtheorem{satz}{Satz}[subsection]
\newtheorem{lem}[satz]{Lemma}
\newtheorem{theo}[satz]{Theorem}
\newtheorem{kor}[satz]{Korollar}
\newtheorem{defi}{Definition}
\theoremstyle{definition}
\newtheorem{bei}[satz]{Beispiel}
\newtheorem{bem}[satz]{Bemerkung}

\graphicspath{ {./images/} }

\begin{document}
	\definecolor{gold}{rgb}{0.9,0.9,0}
	\begin{titlepage}
		\vspace*{-3cm}
		\noindent
		\hspace*{1cm}
			\begin{center}
				\centering
				{\LARGE Mathe LK Zusammengefasst}
			\end{center}
		\begin{center}
		\Large{ Funktionen und Analysis Zusammengefasst}\\[0.5cm]
		\normalsize{von}\\[0.25cm]	
		\large{Benno Schörmann}\\[0.5cm]
		\end{center}
	%\begin{flushleft}
	%\hyperref[subsec:thema1]{\textbf{\large Thema der Facharbeit:}}  \\
	%\end{flushleft}
	Eine gesamte zusammenfassung des Temenbereiches der Analysis und Grundlagen des arbeitens mit Funktionen
	\vfill
	\vfill
	\begin{tabular}{r}
	\Large{28.12.2022}\normalsize
	\end{tabular}
	\hfill
	\quad \\[1.5cm]
	\noindent 
	\renewcommand{\arraystretch}{1.4}
	\end{titlepage}
	\newpage
	\thispagestyle{empty}
	\tableofcontents
	\newpage
	%\listoffigures
	%\thispagestyle{empty}
	%\newpage	
	%\section{Einleitung}
	%(Die Einteilung steht übrigens noch absolut nicht fest, habe sie gestern abend im Halbschlaf angefertigt) \\\\
	%In dieser Facharbeit werde ich über Den Fundamentalsatz der Analysis und die dazugehörigen Nebenpunkte schreiben.
	
	
	
	\section{Funktionen als mathematische Modelle}
		\subsection{I Notation von Funktionen}
			Die Notation, die wir benutzen sieht so aus:
			\begin{equation*}
				f(x) = x
			\end{equation*}
			, wobei $f(x)$ der der korrospondierende y-wert zu dem eingegebenen $x$-wert ist.\\ \\
			Wenn man den y-wert zu einem gegebenen $x$-wert bekommen möchte, dann schreibt man im Beispiel $f(x)=x$ und $x=4$ folgendes:
			\begin{equation}
				f(4) = 4
			\end{equation}
			Man erreicht dies, indem man den nummerischen wert für $x$ auf der seite gegenüber des $f(x)$ einsetzt. Bei zum Beispiel $f(x)=3x-5x$ und $x=3$ rechnet man folgendes:
			\begin{equation*}
				f(3)=3*3-5*3
			\end{equation*}
			\begin{equation*}
				f(3)=-6
			\end{equation*}
			
	
	
	\section{Riemann-Integral}	


\end{document}